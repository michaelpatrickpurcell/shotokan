Begin your kata at the \emph{kiten} (\scalebox{1.5}{\tikz{\pic {kiten={black}}}}).

\begin{tabular}{@{\hskip 0pt}r@{\hskip 1.5ex}l@{\hskip 2ex}c}
1. & Hidari gedan barai & \scalebox{0.5}{\tikz{\pic {gedan_burai2a={black}}}}\\[0.375ex]
2. & Migi chudan oi-zuki & \raisebox{0.5ex}{\scalebox{0.5}{\tikz{\pic {chodan_oi_zuki2a={black}}}}}\\[0.375ex]
3. & Migi gedan barai & \scalebox{0.5}{\tikz{\pic[xscale=-1] {gedan_burai2a={black}}}}\\[0.375ex]
4. & Hidari chudan oi-zuki & \raisebox{0.5ex}{\scalebox{0.5}{\tikz{\pic[xscale=-1] {chodan_oi_zuki2a={black}}}}}\\[0.375ex]
5. & Hidari gedan barai & \raisebox{-0.5ex}{\scalebox{0.5}{\tikz{\pic[rotate=-90] {gedan_burai2b={black}}}}}\\[0.375ex]
6. & Migi chudan oi-zuki & \raisebox{-0.5ex}{\scalebox{0.5}{\tikz{\pic[rotate=-90] {chodan_oi_zuki2b={black}}}}}\\[0.375ex]
7. & Migi gedan barai & \raisebox{-0.5ex}{\scalebox{0.5}{\tikz{\pic[rotate=90, yscale=-1] {gedan_burai2b={black}}}}}\\[0.375ex]
8. & Hidari chudan oi-zuki & \raisebox{-0.5ex}{\scalebox{0.5}{\tikz{\pic[rotate=90, yscale=-1] {chodan_oi_zuki2b={black}}}}}\\[0.375ex]
\end{tabular}
%
%\vspace{0.3ex}
%
%\textbf{Notes}: The triangles for steps 11 and 16 are the same as those for steps 8 and 3/19. Steps 17 \textendash{} 20 are the same as steps 1 \textendash{} 4.

\subsection*{Tips \& Tricks}
Notice that this kata requires you to execute only two fundamental techniques: gedan barai (\scalebox{0.5}{\tikz{\pic {gedan_burai2a={black}}}}) and chudan oi-zuki (\raisebox{0.5ex}{\scalebox{0.5}{\tikz{\pic {chodan_oi_zuki2a={black}}}}}). 

Also, in this kata you alternate between the two techniques. In particular, each gedan barai (\scalebox{0.5}{\tikz{\pic {gedan_burai2a={black}}}}) is followed by chudan oi-zuki (\raisebox{0.5ex}{\scalebox{0.5}{\tikz{\pic {chodan_oi_zuki2a={black}}}}}) in the same direction.

So, one way to improve your performance is to first practice each technique in isolation. Then, practice combining the two techniques. Finally, practice performing the complete kata.

